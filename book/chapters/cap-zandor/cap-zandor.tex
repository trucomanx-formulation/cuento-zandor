\cleardoublepage
\newpage
\ThisULCornerWallPaper{1.0}{chapterimage.eps}
\chapter{Zandor}

Após a morte de Zorra, toda a família ficou muito triste, pois a pesar de suas más costumes, ela sempre foi uma cadela muito amorosa com todos nós e sentíamos sua falta como a de um membro da família. 
Assim, passamos muito tempo chorando, sobretudo minhas irmãs e eu, que ainda eramos crianças. 
Muitas vesses vi a Teodosia ---  minha irmã mais velha, a qual carinhosamente chamávamos ``Tulaco'' ---, iniciar a chorar em silencio ao ver o lugar onde Zorra dormia; inclusive Diofelia  --- ``Dio'', a minha irmã mais nova ---, a pesar de sua curta idade, já sabia distinguir à morte e a ausência que esta deixa. Eu também chorava, talvez mais que elas, porque Zorra era minha companheira fiel, ela me seguia a todos os lugares aonde eu ia; pois, ao ser o filho homem da casa eu tinha que sair a trabalhar à chacra com meu pai, e Zorra sempre fazia mais alegres e memoráveis esses trabalhos.

Nosso único conforto era que tínhamos a Zandor; amávamos ele por ser o último presente que Zorra nos deixou.
Eu olhava para Zandor, todo pequeno e pretinho, e achava muito lindo a ele; sua presença me dizia que de alguma forma, uma parte de Zorra ainda estava conosco. 
Assim, Zandor cresceu sendo criado com muito cuidado e carinho por todos nós;
entretanto, a pessoalidade de Zandor era muito distinta da pessoalidade da sua mãe, ele era um cachorro muito honrado, e era evidente para nós que não tinha nenhuma das más costumes de sua mãe. 
Desde pequeno eu levei a Zandor a todas as minhas caminhadas pela serra; como minha família tinha vacas eu tinha que ir a dar comida e atender elas, e comumente recompensava a Zandor por sua companhia, dando-lhe leite fresco que eu mesmo colhia para matar a nossa sede. 
Com todos esses cuidados, em pouco tempo Zandor virou um cachorrinho forte e muito brincalhão,
que me ajudava a cuidar as vacas, assustava aos pássaros que vinham a comer as sementes na chacra; quando eu gritava seu nome, ele vinha correndo e se parava frente a mim com a marcialidade de um soldado diante do seu general, ele era tão inteligente quanto uma pessoa, e bem mais obediente que eu mesmo.

Um dia, quando estava no campo com Zandor, observamos que uma perdiz saia voando desde uns arbustos; para Zandor, que ainda era filhote, essa foi a primeira vez que ele viu a uma perdiz; eu, pelo contrário, já tinha experiencia com essas aves e sabia que próximo a esse lugar acharíamos um ninho, ovos ou crias. 
Automaticamente gritei --- Zandor! Vamos a buscar ovos! --- Ele latiu ao sentir minha empolgação e avançou junto a mim na direção que indiquei para ele. 
Era gostoso ver a Zandor, pequeno, porém corajoso, batendo seu rabinho, cheirando para todos os lados, levantando e recolhendo suas orelhas; como se, nessa sua primeira missão de busca, quisera demonstrar sua eficácia usando ao máximo todos seus sentidos. 
Só procuramos uns minutos, e de repente os vimos, --- olha Zandor, ovos! --- Gritei, ele latiu como afirmando minha exclamação, e recolhi todos os ovos; Zandor não pegou nenhum dos ovos, ele só me olhava contente enquanto eu colocava eles num saquinho de tecido que usei para levá-los a casa. 

Este procedimento virou comum, e cada vez que eu saia a procurar as vacas, também aproveitava para procurar ovos de perdiz com Zandor; quando os achávamos, eu os levava muito contente a minha mãe; de modo que, todos os dias nós voltávamos com 8, 12 ou até 15 ovos. 
Com o passar dos meses Zandor virou um especialista em encontrar ovos, além de que ele já não era mais um filhote, e eu não precisava acompanhar ele; assim, enquanto eu trabalhava ele saia por conta própria a procurar ovos; no instante que ele os achava, latia para mim, repetidas vezes e sem descanso, ate chamar minha atenção; de modo que minha única missão era recolher eles e leva-los a casa.
Algumas vezes cozinhávamos os ovos, outras vezes os fritávamos; numa dessas ocasiões, meu pai chegou à casa enquanto estávamos cozinhando os ovos,--- e esse ovo?--- perguntou ele, e nós contentos e cheios de orgulho respondemos,--- Zandor achou!---
Ele meditou um pouco e replicou, --- Zandor achou ... e que coisa deram a ele?--- A pergunta nos pegou de surpreso e falamos em voz baixa --- nada pai, só a comida da casa... --- Meu pai nos olhou e nos indicou calmadamente --- se ele foi quem achou, ele também deve participar ---; 
nesse momento meu pai pegou um ovo crú e deu para ele, Zandor pegou contente u ovo e chupou ele até deixar só a casca; a partir de ali Zandor se acostumou a comer eles. Assim, virou uma tradição que cada vez que ele encontrava ovos, eu os levava à casa, os entregava a minha mãe e ela na sua vez entregava dois a Zandor; porém, nunca lhe dávamos os ovos quando os encontrava, só na casa, ele por sua parte sabia esperar e nunca pegou eles, só esperava paciente o momento que minha mãe entregue a ele seus dois ovos e ia contento a seu cantinho a comer eles.


Para mim Zandor era maravilhoso, a qualquer lugar que ia ele me acompanhava, quando estava triste ele se sentava a meu lado, e ate chorava comigo, fazendo um sonido agudo, que eu sentia cheio de solidariedade. 
Por outro lado, se ele percebia que eu estava contente, levantava suas orelhinhas e iniciava a pular e correr de um lado a outro; assim, durante um bom tempo andamos e crescemos juntos, o tempo passou, e eu cumpri oito e nove anos.

Um dia meu pai decidiu abater um boi; la na serra não se mata um boi sem nenhum motivo, só em ocasiões importantes como festas regionais, casamentos ou semelhantes; porém, eu sabia que nessa época não tínhamos nenhuma festividade, e pensava que a meu pai simplesmente ocorreu-lhe abater o boi sem nenhum motivo, porém, não era assim. 
Eu tinha um irmão mais velho que morava na capital, em Lima, o qual eu não conhecia, só sabia de sua existência, pois meus pais sempre falavam dele e se referiam a ele como ``Seve'', pelo que nessa época eu achava que ele se chamava dessa forma, porém, seu nome não era esse e sim Severino; eu não o conhecia porque ele viajou a Lima quando eu era muito pequeno, seguramente eu tinha visto ele com anterioridade, quando bebê, mas eu não tinha lembrança disso. 
Assim, meu pai e minha mãe tinham a intenção de fazer charque para mandar-lho como presente a meu irmão, e com esse fim abateram o boi e prepararam sua carne com sal.


Nosso lar estava composto por uma casa pequena e uma casa grande, separadas entre si; perto da pequena tínhamos duas figueiras grandes; meu pai usou uma dessas figueiras para amarrar uma corda até a casa pequena, e sobre ela pendurou a carne para que se seque com o sol, porém, meu pai decidiu pendurar no tronco da figueira uma perna que pesava muito.
Lá a costume é que de noite a carne é recolhida e levada para a casa, com o proposito que os animais da noite não venham a comer ela, de modo que ao dia seguinte de manhã, com os primeiros raios de sol, a carne era novamente pendurada; porém, esse dia meu pai recolheu só o charque que estava pendurado na soga e esqueceu a perna que tinha colocado na figueira. 
O pior foi que ali, onde meu pai deixou a carne, tranquilamente qualquer cachorro, zorro, u outro animal da serra, poderia pegar ela facilmente, sem a necessidade de pular, pois, não estava a muita altura.
Nessa noite Zandor não parou de latir; nos desde a casa grande só escutávamos seu barulho com curiosidade, pois nenhum membro da família lembrou da perna pendurada na figueira; pelo barulho só reconhecíamos que algumas vezes chegavam outros cachorros, outras vezes não escutávamos nenhum outro animal, só a Zandor latindo com muita força; meu pai, bravo pelo ruído, só falava --- que coisa é que quer esse cachorro, que não nos deixa dormir! --- Porem, ele não saia da casa grande a indagar, eu tinha muito medo de toda essa situação, pois la na serra se contam historias dos seres que habitam na noite. 
Alguns dizem que de noite anda o ``cuco''\footnote{Também chamado, coca ou coco, este é um ser mítico, uma espécie de fantasma, bruxa ou bicho-papão que anda de noite pelos caminhos.}, e as crianças tínhamos um terror extremo a esse ser; para piorar a situação meu pai tinha a costume de contar histórias de suas viagens, de como de noite achou o cuco nos caminhos da serra, ou também que em algum povo perto o cuco tinha matado a algum vizinho, que tinha chupado a sangue de outro ou simplesmente assustado a algum outro caminhante noturno; devo reconhecer que apesar do terror que me causavam as historias do meu pai, eu gostava delas e de passar medo escutando-as; ele sempre me contavas suas aventuras de quando saia a trabalhar nas distintas cidades e das coisas que veia, dos problemas que aconteciam no caminho e dos personagens que apareciam quando ele se deslocava a pé.

Por exemplo, um dia meu pai me contou que quando estava viajando de nosso povo ate ``Cangallo'', um povo vizinho, no médio do caminho a noite o atrapou, e ele iniciou a procurar entre os trilhos alguma casa que pudesse dar pousada; enquanto ele estava nessa tarefa, ele escutou um pássaro ao qual nós na serra chamamos: ``huaychao'', cujo canto ou grito é de mau augúrio, pois só e feito de noite.
Meu pai falava que quando o ``Huaychao'' cantava era porque o mal estava perto, que ele canta porque viu o mal andando, talvez na forma de algum ``jarjacha''\footnote{Também chamado carcaq ou qarqacha.}; para nós os jarjachas são seres da noite, são pessoas que se levantam dos seus túmulos, pois ao ter feito coisas terríveis em vida estão condenados a não morrer e vagar de noite entre o sofrimento e a ira. 
Então, meu pai sempre me advertia muito serio, que se na noite escutava o huaychao devia ter cuidado porque um jarjacha estava perto. 
Assim, quando meu pai escutou o huaychao, iniciou a correr pulando pedras e atravessando riachos ate que só e assustado achou uma casa; rapidamente tocou a porta e desde dentro escutou uma voz de mulher que lhe perguntava qual era seu problema; meu pai todo assustado tentou lhe explicar que a noite tinha chegado em médio do seu caminho e que precisava só um lugar para dormir; a senhora desde o interior lhe respondeu que, da mesma forma que ele falou, seres que não são pessoas, cucos, andam pela noite enganando aos moradores para conseguir entrar a suas casas, --- de repente você é um deles --- falou ela, e negou a meu pai um lugar para dormir; ele insistia com a voz tremida pelo medo, pois  sabia que tudo isso era verdade, pois já tinha escutado ao huaychao e sabia que o mal estava perto; por fim, após muito insistir, a senhora se comoveu e o deixou entrar à casa. 
A senhora, toda curiosa pela situação, perguntou a meu pai porque andava de noite, e ele explicou que só estava tentando ir de Occo a Cangalho, porém, teve problemas no caminho e a noite ganhou ele; imediatamente a senhora respondeu e tom maternal --- Porque você anda de noite! Só ontem um jarjacha comeu a uma pessoa, agora esse vizinho está morto, hoje mesmo enterramos ele ---.


Por todas essas historias, sair da casa grande de noite, só porque o cachorro latia, era uma completa temeridade; até meu pai tinha medo de sair, ele só gritava para Zandor desde dentro da casa, segurando sua ``guaraca''\footnote{Corda muito versátil que pode ser usada como cinto de calças ou para castigar crianças desobedientes.}, golpeando com ela a parede. 
Os demais membros da família só escutávamos resignados, intentando dormir apesar do barulho.
Assim, a noite passou, e praticamente nenhum de nós conseguiu dormir; quando os primeiros raios do sol tocaram nossa janela, todos nós saímos em direção da casa pequena, e para nossa surpresa vimos a perna de boi pendurada na figueira, para nós foi evidente que de noite os animais do campo tinham chegado a comer a carne e Zandor tinha defendido ela, brigando, latindo e sem dormir. 
Ele estava encolhido em forma de bolinha ao lado da perna de boi, e ao vernos chegar só olhou para nós com um olhar cansado; meu pai se admirou pelo desempenho de Zandor, pois a carne estava intata, --- como me olvidei a perna! Por isso chegavam os cachorros! --- 
Exclamou meu pai, automaticamente entrou à casa, pegou uma faca, cortou com ela um pedaço grande de carne da perna do boi, ainda pendurada na figueira, e a entregou a Zandor como um prêmio; só nesse momento ele olhou a carne com vontade, pegou ela e foi a seu canto a comer.

Assim, cresceu Zandor, sendo sempre a costume dele ser honrado, se você não dava ele não comia nada. 
No campo os cachorros comem bem, eles são bem cuidados, comem a mesma comida dos donos de casa e são tratados com carinho, como um membro mais da família.

