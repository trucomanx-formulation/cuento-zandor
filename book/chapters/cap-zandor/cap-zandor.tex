\cleardoublepage
\newpage
\ThisULCornerWallPaper{1.0}{chapterimage.eps}
\chapter{Zandor}

Após a morte de Zorra, toda a família ficou muito triste, pois a pesar de suas más costumes, ela sempre foi uma cadela muito amorosa com todos nós e sentíamos sua falta como a de um membro da família. 
Assim, passamos muito tempo chorando, sobretudo minhas irmãs e eu, que ainda eramos crianças. 
Muitas vesses vi a Teodosia ---  minha irmã mais velha, a qual carinhosamente chamávamos ``Tulaco'' ---, iniciar a chorar em silencio ao ver o lugar onde Zorra dormia; inclusive Diofelia  --- ``Dio'', a minha irmã mais nova ---, a pesar de sua curta idade, já sabia distinguir à morte e a ausência que esta deixa. Eu também chorava, talvez mais que elas, porque Zorra era minha companheira fiel, ela me seguia a todos os lugares a onde ia; pois ao ser o filho homem da casa eu tinha que sair a trabalhar à chacra com meu pai, e Zorra sempre fazia mais alegres e memoráveis esses trabalhos.

Nosso único consolo era que tínhamos a Zandor; amávamos ele por ser o último presente que Zorra nos deixou.
Eu olhava para Zandor, todo pequeno e pretinho, e achava ele muito lindo, e sua presença me dizia que de alguma forma, uma parte de Zorra ainda estava com nós. 
Assim, Zandor cresceu sendo criado com muito cuidado e carinho por todos nós;
entretanto, a pessoalidade de Zandor era muito distinta da pessoalidade da sua mãe, ele era um cachorro muito honrado, e era evidente para nós que não tinha nenhuma das más costumes de sua mãe. 
Desde pequeno eu levei a Zandor a todas minhas caminhadas pela serra; como minha família tinha vacas eu tinha que ir a dar comida e atender elas, e comumente recompensava a Zandor por sua companhia, dando-lhe leite fresca que eu mesmo colhia para matar nossa sede. 
Com todos esses cuidados, em pouco tempo Zandor virou uma cachorrinho forte e muito brincalhão,
que me ajudava a cuidar as vacas, assustava aos pássaros que vinham a comer as sementes na chacra, quando eu gritava seu nome ele vinha correndo, e se parava frente a mim com a marcialidade de um soldado diante do seu general, ele era tão inteligente quanto uma pessoa, e bem mais obediente que eu mesmo.

Um dia quando estava no campo com Zandor, observamos que uma perdiz saia voando desde uns arbustos; para Zandor, que ainda era filhote, essa foi a primeira vez que ele viu a uma perdiz; eu, pelo contrario, já tinha experiencia com essas aves e sabia que próximo a esse lugar acharíamos um ninho, ovos ou crias. 
Automaticamente gritei --- Zandor! Vamos a buscar ovos! --- Ele latiu ao sentir minha empolgação e avançou junto a mim na direção que indiquei para ele. 
Era gostoso ver a Zandor, pequeno porém corajoso, batendo seu rabinho, cheirando para todos lados, levantando e recolhendo suas orelhas; como se, nessa sua primeira missão de busca, quisera demonstrar sua eficacia usando ao máximo todos seus sentidos. 
Só procuramos uns minutos, e de repente os vimos, --- olha Zandor, ovos! --- Gritei, ele latiu como afirmando minha exclamação, e recolhi todos os ovos; Zandor não pegou nenhum ovo, ele só me olhava contente enquanto eu colocava eles num saquinho de tecido que usei para levar eles a casa. 
Este procedimento virou comum, e cada vez que eu saia a procurar as vacas, também procurava ovos de perdiz com Zandor; quando os achávamos eu os levava muito contente a minha mãe; assim, todos os dias nós voltávamos com 8, 12 ou até 15 ovos. 
Com o passar dos meses Zandor virou um especialista em achar ovos, além de que ele já não era mais um filhote, e eu não precisava acompanhar ele; assim, enquanto eu trabalhava ele saia por conta própria a procurar ovos; no instante que ele os achava, latia para mim repetidas vezes e sem descanso ate chamar minha atenção; de modo que minha única missão era recolher os ovos e levar eles a casa.
As vezes cozíamos os ovos, outras vezes fritávamos eles; numa dessas ocasiões, meu pai chegou a casa enquanto estávamos cozinhando os ovos,--- e esse ovo?--- perguntou ele, e nós contentos e cheios de orgulho respondemos,--- Zandor achou!---
Ele meditou um pouco e replicou, --- Zandor achou ... E que coisa deram a ele?--- A pergunta nos pegou de surpreso e falamos em voz baixa --- nada pai, só a comida da casa... --- Meu pai nos olhou e nos indicou calmadamente --- se ele foi quem achou, ele também deve participar ---. 
Nesse momento meu pai pegou um ovo crú e deu para ele, Zandor pegou contente o ovo e chupou ele ate deixar a cascara; a partir de ali Zandor se acostumou a comer ovo; assim, virou uma tradição que cada vez que ele encontrava ovos, eu os levava à casa, os entregava a minha mãe e ela na sua vez entregava dois a Zandor; porém, nunca lhe dávamos os ovos quando os encontrava, só na casa, ele por sua parte sabia esperar e nunca pegou eles, só esperava impaciente o momento que minha mãe entregue seus dois ovos e ia contento a seu cantinho a comer eles.


Para mim Zandor era maravilhoso, a qualquer lugar que ia ele me acompanhava, quando estava triste ele se sentava a meu lado, e ate chorava comigo fazendo um sonido agudo, que eu sentia cheio de solidariedade. 
Por outro lado, se ele percebia que eu estava contente, levantava suas orelhinhas e iniciava a pular e correr de um lado a outro; assim, durante um bom tempo andamos e crescemos juntos, o tempo passou, e eu cumpri oito e nove anos.
Por essa época meu pai decidiu abater um boi, la na serra não se mata um boi sem nenhum motivo, só em ocasiões importantes como festas regionais, casamentos; porém, eu sabia que nessa época não tínhamos nenhum evento importante, e eu pensava que a meu pai simplesmente  ocorreu-lhe abater sem motivo nenhum; porém não era assim, eu tinha um irmão mais velho que morava na capital, em Lima, eu não conhecia a meu irmão, só sabia de sua existência porque meu pai se referia a ele como ``Seve'', pelo que eu achei que ele se chamava Seve, porém seu nome não era esse e sim Severino, eu não conhecia ele porque ele viajou a Lima quando eu era muito pequeno, seguramente eu tinha visto com anterioridade a ele quando bebê, mas não tinha lembrança disso; porém meus pais sempre falavam dele. 
Assim, meus pais tinham a intenção de fazer charque com a carne do boi para mandar a meu irmão 


