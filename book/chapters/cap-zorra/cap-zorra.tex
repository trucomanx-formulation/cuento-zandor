
\chapter{Zorra}
Quando era criança, minha família tinha uma cadela que se chamava Zorra, ela era de caráter gentil, e meu pai e eu íamos com ela a todos os lados; quando pelos caminhos nós avistávamos a algum conhecido ou familiar, eles gritavam --- bom dia Don-Juande! --- E nos respondíamos suas saudações com a mesma alegria e energia.
Meu pai na verdade se chamava Juan de Dios (João de Deus); porém, carinhosamente, todo mundo preferia chamar ele Don-Juande.

Eu provavelmente teria seis anos nessa época, e lembro a minha cadela correndo de forma ágil adiante de nós, abrindo-nos o caminho por meio dos montes, ladrando e sorrindo.
Estas caminhadas eram muito comuns, pois tínhamos que ir a levar comida a nossa vaca, a procurar ela sim se perdia, ou a dar manutenção à chacra.

Num princípio, eu não tinha percebido que minha cadela tinha algumas más costumes, pois com ela nós convivíamos e andávamos de dia, e seu comportamento era irrepreensível. 
Ainda lembro a primeira vez que fui ao rio, com ela e meu pai, com a finalidade de obter alguns peixes para serem fritos no almoço; eu decidi acompanhar eles, pois gostava de sair a andar. 
Para chegar ao rio tivemos que descer uma ladeira com um caminho cercado de plantas de figo das índias, e as margens das águas estavam cheias de juncos, enquanto meu pai pescava eu e minha cadela procurávamos ovos de perdiz; porem, sempre ela achava primeiro, comia tudo, ou quase tudo, de modo que só podia salvar alguns para mim.


Dessa forma passaram alguns anos, onde ninguém chegou a casa trazendo alguma reclamação ou comentário sobre ela; porém, um dia entrei na minha horta e  fiquei surpreso ao achar varias coisas dentro de um esconderilho. Havia uma sacola de tecido com pão, outra com açúcar, outra com doces, etc. 
Para mim tudo isso era inacreditável, pois nós só tínhamos coisas como açúcar ou bolachas, quando meu pai voltava de suas viagens após trabalhar quatro meses em Ica, ou alguma outra cidade grande.
Num primeiro momento, a alegria invadiu meu coração, mas lembrei que meu pai era uma pessoa muito rigorosa, não gostava pegar as coisas dos outros, e dizia --- se você encontra alguma coisa no caminho, no campo, ou na pampa, você não deve pegar ---, e ele agregava, --- seguramente alguém deixou cair, a pessoa que perdeu vai voltar a procurar, e se você leva, ela não vai encontrar ---, 
eu lembrava muito bem desse ensinamento, pois uma vez quando estava na solidão do campo, pastando a minha vaca, achei uma ferramenta para fazer fios de lã, que no meu povo chamávamos de callapa; provavelmente a ferramenta era de algum outro pastor que passou por ali, mas nesse momento não pensei nisso, só peguei ela e me dirigi a minha casa; já na tarde, cheguei  muito contente, falando --- Mamãe, papai, olha o que achei no campo! --- 
Meu pai imediatamente respondeu --- Aqui não tem nada para ser encontrado! Isso é de alguém, alguma pessoa perdeu e vá ir a procurar, anda e deixa isso no lugar que você achou ---. 
Nesse momento um frio desceu desde a ponta da minha cabeça até as pontas dos meus pés, pois já eram quase as seis da tarde, todo estava obscuro, as poucas luzes eram muito distantes e chegavam das casas dos vizinhos, --- pois la no povo, as famílias moravam em casas que estavam separadas, uma da outra, por uma grande distância, 200, 300, 400 metros e algumas vezes até mais  ---, e para finalizar, eu era muito medroso no que se refere a lugares obscuros e a onde tinha que ir estava muito longe.
Diante da ordem do meu pai, eu fui correndo e chorando nessa direção, no caminho quase não podia distinguir as coisas a uns metros de mim, pois não tínhamos lua nessa noite; porém, quando já estava quase na metade do caminho, observei a meu redor e numa trilha paralela à minha, entre pedras e árvores grandes, meu pai me seguia a escondidas e a uma distância significativa.
Com essa certeza no meu coração eu segui meu caminho com um pouco mais de tranquilidade, pois sentia que meu pai estava me cuidando; mesmo assim, eu seguia chorando, pois, na serra a noite é absoluta e densa, e os sons do caminho alimentavam facilmente a imaginação de uma criança.
No último terço do caminho eu decidi ir correndo, pois, sentia que já não podia aguentar mais tempo essa situação; por fim cheguei ao lugar e deixei a callapa lá.
Na volta também senti a presença do meu pai uns metros atrás de mim, ou pelo menos isso queria acreditar eu, e cheguei a minha casa em menos tempo do que gastei para ir, e quando entrei meu pai estava sentado lá, como se nada houvesse acontecido.


Por essa velha experiência, e tendo a certeza da autoria de zorra, pois esse era um dos seus lugares favoritos, eu tive muito medo por ela; pois sabia que meu pai não ia gostar que Zorra estivesse pegando as coisas de outras pessoas; pelo que se avisava a meu pai, ele iria castigar ela severamente; por esse motivo decidi não falar nada sobre minha descoberta, porém, tenho que reconhecer que, além do amor a minha cadela, também pesou na minha decisão que o lugar estivesse cheio de açúcar, pão e outras coisas, que para nós, gente da serra, eram luxos. Por esse motivo, eu iniciei a passar por ali antes de ir à escola ou a chacra, para comer bolachas, água com açúcar, ou qualquer outra coisa boa que estivesse por ali. 
Para meu desespero, um dia Zorra trouxe demasiadas coisas, não sei de onde pegou elas, pois, ela só trabalhava de noite, enquanto as pessoas da casa dormiam. Assim, meu pai descobre a situação diante desse aumento na criminalidade, que desbordava seu esconderilho.
Muito para meu pesar, ele castigou severamente a Zorra, desde minha casa eu só escutei seus lamentos recebendo a punição, pois eu preferi não ver.
Depois dessa vez, ela deixou de  levar as coisas à horta, e o problema parecia resolvido, porém, logo descobriríamos que longe da casa, numa pedra grande e obscura, perto da casa de um vizinho, Zorra havia reiniciado suas atividades; assim, de dia, diante os olhos da família, se comportava com uma cadela exemplar, porém de noite roubava as mais variadas coisas dos vizinhos.
Nesse momento foi a primeira vez que uma pessoa chegou a minha casa a fazer uma reclamação, o pagante denunciou que ele tinha visto em pessoa como Zorra havia entrado na sua casa a roubar, a indignação do meu pai foi tão grande quanto sua vergonha, pois não foi um vizinho familiar nosso, que poderia entender a situação, se não que foi uma vítima que morava muito longe de nós, em outro povo, que foi perguntando entre os vizinhos até achar nossa casa.
Nessa ocasião meu pai castigou mais duramente a Zorra; diante essa difícil situação, minha maior tristeza era que eu já compreendia que esse problema não ia a solucionar-se com outro castigo, e minhas dúvidas foram confirmadas quando percebi que ela seguia saindo de noite. 
Muito tempo depois descobriríamos que Zorra novamente tinha mudado de esconderilho, e que levava suas coisas em outra pedra grande, na casa de uma vizinha que era viúva e que carinhosamente eu chamavamos avó Mesla; --- na verdade, ela não era familiar meu, mas a costume na serra era chamar avó a qualquer pessoa de idade, como sinal de respeito, pois vivemos com eles como se fosse nossa família ---.
Porém, antes que alguma pessoa da minha casa conhece-se esse novo lugar onde Zorra ocultava seus objetos roubados, outra pessoa chegou a denunciar novamente à cadela; e mesmo que meu pai castigou, gritou e tentou seguir ela, não conseguimos achar onde ela escondia as coisas; assim, o tempo transcorreu sem que essa incógnita fosse resolvida. 

Um dia, meu pai, meus irmãos e eu decidimos descer ao rio para pescar, porém, esse dia não achamos a Zorra para que nos acompanhe, quando chegou a hora das quatro ou cinco da tarde, quando estávamos prontos a regressar a casa com a pesca do dia, escutei um ruído entre os juncos do rio, onde costumava brincar com Zorra, por pura curiosidade me acerquei a averiguar que era esse ruido como um choro agudo; para minha surpresa era um cachorrinho, pequeno e pretinho, que minha cadela havia parido.  
Para que meu pai não veja ele, eu escondi o filhote dentro do meu agasalho, pois desconfiava que ele me deixa-se levar um novo cachorro a casa, dado que os problemas que gerava Zorra já eram suficientes para nós, como para arriscar mais a reputação da família com outro cachorro.

Só quando cheguei a casa tireo o cachorro de dentro das minhas roupas, e diante dos meus irmãos, minha mãe e meu pai, presentei o novo membro da família. Como nessa época, minhas irmãs estavam aprendendo a ler usando um livro chamado "Lola e Pepe", onde nas suas histórias descreviam a um cachorro chamado Zandor, eu decidi usar esse nome para meu novo cachorro. 
Mesmo com seus novos deveres de mãe, Zorra não deixava de causar problemas, pois a nossos ouvidos chegavam histórias de roubos em povos distantes, e as ausências de Zorra se volviam cada vez mais prolongadas.
Para nossa má sorte, num povo vizinho, se tinha espalhado a notícia que era uma cadela a responsável de todos os roubos, e num triste dia, as pessoas se organizaram, conseguiram encurralar e atrapar ela, e finalmente, deram morte a minha querida Zorra.
Esse mesmo dia chegou a minha casa a notícia da sua morte, entre lágrimas e lamentos fomos a esse povo para enterrar a nossa cadela, pois nos informaram que eles simplesmente tinham jogado seu corpo na beira do rio; quando chegamos lá, choramos até que nossas lágrimas se secaram, pois, ela tinha sido nossa amiga fiel, e mesmo que soubéssemos das suas más costumes, nós amávamos ela.
Nesse mesmo lugar, ao finalizar o dia e com o rio como testemunha, fizemos uma pequena cerimônia, e enterramos a quem em vida foi conhecida por nós como Zorra.

Ao dia seguinte, a notícia tinha se espalhado no meu povo, e mesmo que nós assumimos que não receberíamos nenhuma condolência de parte dos vizinhos, a avó Mesla chegou à porta de nossa casa chorando, e entre seus lamentos dizia:
\begin{quotation}
Eu não tinha panela, e Zorra me levou uma panela; 
eu não tinha frigideira, e Zorra me levou uma frigideira; 
eu não tinha colher, e Zorra me levou uma colher;
eu não tinha faca, e Zorra me levou uma faca;
quando tive fome, Zorra me levou pão ...  
\end{quotation}
Minha mãe abraçava ela, enquanto avó Mesla continuava com sua litania, por aquele animal que, a seu modo de ver, só tinha chegado a sua vida para ajudar ela no seu momento de maior necessidade.
 

