
\chapter{Zorra}
Quando era criança, minha família tinha uma cadela que se chamava Zorra, ela era de caráter gentil, e meu pai e eu íamos com ela a todos os lados; quando pelos caminhos nós avistávamos a algum conhecido ou familiar, eles gritavam --- bom dia Don-Juande! --- E nos respondíamos suas saudações com a mesma alegria e energia.
Meu pai na verdade se chamava Juan de Dios (João de Deus); porém, carinhosamente, todo mundo preferia chamar ele Don-Juande.

Eu provavelmente teria seis anos nessa época, e lembro a minha cadela correndo de forma ágil adiante de nós, abrindo-nos o caminho por meio dos montes, ladrando e sorrindo.
Estas caminhadas eram muito comuns, pois tínhamos que ir a levar comida a nossa vaca, a procurar ela sim se perdia, ou a dar manutenção à chacra.

Num princípio, eu não tinha percebido que minha cadela tinha algumas más costumes, pois com ela nós convivíamos e andávamos de dia, e seu comportamento era irrepreensível. 
Ainda lembro a primeira vez que fui ao rio, com ela e meu pai, com a finalidade de obter alguns peixes para serem fritos no almoço; eu decidi acompanhar eles, pois gostava de sair a andar. 
Para chegar ao rio tivemos que descer uma ladeira com um caminho cercado de plantas de figo das índias, e as margens das águas estavam cheias de juncos, enquanto meu pai pescava eu e minha cadela procurávamos ovos de perdiz; porem, sempre ela achava primeiro, comia tudo, ou quase tudo, de modo que só podia salvar alguns para mim.


Dessa forma passaram alguns anos, onde ninguém chegou a casa trazendo alguma reclamação ou comentário sobre ela; porém, um dia entrei a meu jardim e fiquei surpreso ao achar varias coisas, havia pão numa sacola de tecido, outra com açúcar, outra com doces, etc. 
Para mim tudo isso era inacreditável, pois nós só tínhamos coisas como açúcar ou bolachas, quando meu pai voltava de suas viagens após trabalhar quatro meses em Ica, ou alguma outra cidade grande.
Num primeiro momento, a alegria invadiu meu coração, mas lembrei que meu pai era uma pessoa muito rigorosa, não gostava pegar as coisas dos outros, e dizia --- se você encontra alguma coisa no caminho, no campo, ou na pampa, você não deve pegar ---, e ele agregava, --- seguramente alguém deixou cair, a pessoa que perdeu vai voltar a procurar, e se você leva, ela não vai encontrar ---, 
eu lembrava muito bem desse ensinamento, pois uma vez quando estava na solidão do campo, pastando a minha vaca, achei uma ferramenta para fazer fios de lã, que no meu povo chamávamos de callapa; provavelmente a ferramenta era de algum outro pastor que passou por ali, mas nesse momento não pensei nisso, só peguei ela e me dirigi a minha casa; já na tarde, cheguei  muito contente, falando --- Mamãe, papai, olha o que achei no campo! --- 
Meu pai imediatamente respondeu --- Aqui não tem nada para ser encontrado! Isso é de alguém, alguma pessoa perdeu e vá ir a procurar, anda e deixa isso no lugar que você achou ---. 
Nesse momento um frio desceu desde a ponta da minha cabeça até as pontas dos meus pés, pois já eram quase as seis da tarde, todo estava obscuro, as poucas luzes eram muito distantes e chegavam das casas dos vizinhos, localizadas a 400 metros ou mais, e para finalizar, eu era muito medroso no que se refere a lugares obscuros e a onde tinha que ir estava muito longe.
Diante da ordem do meu pai, eu fui correndo e chorando nessa direção, no caminho quase não podia distinguir as coisas a uns metros de mim, pois não tínhamos lua nessa noite; porém, quando já estava quase na metade do caminho, observei a meu redor e numa trilha paralela à minha, entre pedras e árvores grandes, meu pai me seguia a escondidas e a uma distância significativa.
Com essa certeza no meu coração eu segui meu caminho com um pouco mais de tranquilidade, pois sentia que meu pai estava me cuidando; mesmo assim, eu seguia chorando, pois, na serra a noite é absoluta e densa, e os sons do caminho alimentavam facilmente a imaginação de uma criança.
No último terço do caminho eu decidi ir correndo, pois, sentia que já não podia aguentar mais tempo essa situação; por fim cheguei ao lugar e deixei a callapa lá.
Na volta também senti a presença do meu pai uns metros atrás de mim, ou pelo menos isso queria acreditar eu, e cheguei a minha casa em menos tempo do que gastei para ir, e quando entrei meu pai estava sentado la, como se nada houvesse acontecido.

Por essa velha experiencia, sabia que meu pai.



